\documentclass[10pt]{article}

\usepackage{biunotation}
\usepackage[margin=2cm]{geometry}
\usepackage{hyperref}
\usepackage{xcolor}

\hypersetup{
    colorlinks=true,
    linkcolor=black,
    urlcolor=blue,
}

\definecolor{beige}{RGB}{250,200,150}

\def\BiU{B\kern-0.1em\lower0.3em\hbox{I}\kern-0.1emU}

\def\com#1{\texttt{\detokenize{#1}\kern-1ex}}
\def\Aparam#1{$\left\{\left<\textsl{#1}\right>\right\}$}
\def\Bparam#1{$\left[\left<\textsl{#1}\right>\right]$}

\def\repolink{https://github.com/ari-feiglin/BIUnotation.git}
\def\refrepo#1{\href{\repolink}{#1}}

\makeatletter
\def\param{\@ifstar\Bparam\Aparam}
\makeatother

\def\showcase#1{\setbox0=\hbox{\tt Scriptscript style:}
    \begin{itemize} 
        \item \leavevmode\hbox to \wd0{\tt Display style:\hfil}       $\displaystyle #1$
        \item \leavevmode\hbox to \wd0{\tt Text style:\hfil}          $\textstyle #1$
        \item \leavevmode\hbox to \wd0{\tt Script style:\hfil}        $\scriptstyle #1$
        \item \leavevmode\hbox to \wd0{\tt Scriptscript style:\hfil}  $\scriptscriptstyle #1$
    \end{itemize}
}

\begin{document}

\setcounter{tocdepth}{3}
\tableofcontents
\newpage

\section{Overview}

\BiU-notation is a package for typesetting unique/esoteric mathematical notation used by lecturers in Bar Ilan University.

The package is intended to be independent, as in it does not require any other package installations and can be used in plain-\TeX (with the removal of La\TeX macros that are necessary for La\TeX packages).

Any suggestions/issues should be made to the \refrepo{github repository}.

\subsection{Installation}

If you're using an online La\TeX editor like \href{https://www.overleaf.com}{overleaf}, then just upload the \href{run:./biunotation.sty}{biunotation.sty} file to your project, and skip to \textbf{Step 3}. The other steps require you to have La\TeX installed locally (or Lyx).

\noindent\textbf{Step 1}:\par
{\advance\leftskip1cm\parindent=0pt
The first step is to somehow download \href{run:./biunotation.sty}{biunotation.sty}. 

This can be done by cloning \refrepo{this repository}. This can be done on the command line (assuming {\tt git} is installed) with \par
\vskip1ex\hfil\colorbox{black}{\textcolor{green}{\$} \textcolor{gray}{\tt git clone \repolink\ /path/to/destination}}\hfil\par\vskip1ex
So for example:\par
\vskip1ex\hfil\colorbox{black}{\textcolor{green}{\$} \textcolor{gray}{\tt git clone \repolink\ $\sim$/Downloads}}\hfil\par\vskip1ex
Will create a directory called {\tt biunotation} in {\tt ~/Downloads} with the repository.\par
}

\noindent\textbf{Step 2}:\par
{\advance\leftskip1cm\parindent=0pt
Now we must move the correct file to the correct directory. First locate {\tt biunotation.sty}. If you cloned the repository into your current directory, it can be found at {\tt ./biunotation/biunotation.sty}.

If you'd like to only use this package once, then move {\tt biunotation.sty} to the same directory as your {\tt .tex} file (or {\tt .lyx} file if you're using Lyx).

If you'd like to use this package multiple times, move it to {\tt ~/texmf} (if it doesn't exist, create it).\par
}

\noindent\textbf{Step 3}:\par
{\advance\leftskip1cm\parindent=0pt
In your {\tt .tex} file, in its preamble (before {\tt \detokenize{\begin}\kern-1ex\{document\}} but after {\tt\detokenize{\documentclass}...}), put 

\hfil\colorbox{beige}{\hbox to 10cm{\tt\textcolor{blue}{\detokenize{\usepackage}\kern-1pt}\{biunotation\}}}\hfil

If you are using Lyx, go to {\tt Document > Settings > ... > LaTeX Preamble} and paste the above command.\par
}

\newpage
\section{Macros}
\subsection{Symbols}
\subsubsection{Mathmode Symbols}

\begin{itemize}

    \item \com{\dcup}: A \textsl{binary operator} that is an alternative method of denoting \href{https://en.wikipedia.org/wiki/Disjoint_union}{disjoint unions} (an alternative is $\sqcup$). \showcase{A\dcup B}
    \item \com{\bigdcup}: The \textsl{large operator} brother of \com{\dcup}. \showcase{\bigdcup_{i=1}^\infty A_i}
    \item \com{\indep}: A \textsl{binary relation}\footnote{This is not defined as a binary relation but should nevertheless be used as one} that is an alternative method of denoting \href{https://en.wikipedia.org/wiki/Independence_(probability_theory)}{independence} (an alternative is $\perp$). \showcase{A\indep B}

\end{itemize}

\subsubsection{Textmode Symbols}

\begin{itemize}

    \item \com{\pp}: (You have a smol one)

\end{itemize}

\newpage
\subsection{Symbol Creation Macros}

These are macros used for creating new symbols.

\begin{itemize}

    \item \com{\putsym}\param{primary}\param{secondary}: Places the \textsl{secondary} symbol over the \textsl{primary} symbol.\footnote{In order to allow you to worry about extra space around \textsl{primary}, it becomes a \textsl{ordinary} math symbol.} 
        \par\leavevmode\hbox{\vbox{\hbox{\colorbox{beige}{\tt \detokenize{$\putsym{\cup}{\cdot}$}}}\hbox{\colorbox{beige}{\tt \detokenize{$\putsym{\triangle}{*}$}}}}\kern10pt\vbox{\hbox{$\putsym{\cup}{\cdot}$}\hbox{$\putsym{\triangle}{*}$}}}
\end{itemize}


\end{document}

